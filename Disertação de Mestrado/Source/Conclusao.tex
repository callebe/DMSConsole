\label{cap:Conclusao}
Ao longo deste trabalho foram apresentados os conceitos fundamentais acerca do funcionamento e funcionalidade do sistema de indicadores de destino, e consequente as vantagens de uma moderniza��o por meio de um novo modelo de console para este sistema. Este novo console dotado de mais recursos de conectividade e intera��o com o usu�rio, desdo \textit{software} ao \textit{hardware} formam aqui desenvolvidos. O \textit{software} atendeu as necessidades do projeto e forneceu novas funcionalidades ao console, mantendo ainda o desempenho da aplica��o. A interface de usu�rio, parte constituinte do \textit{software}, se apresentou fluida, pr�tica e amig�vel, tornando o console ainda mais �til e agrad�vel ao usu�rio do que a antiga vers�o SICON II. J� o \textit{hardware} apresentou-se bastante robusto e compacto, sendo todo o console constitu�do de apenas 3 PCBs; o Raspberry Pi, o m�dulo \textit{Display Touch Screen} e a placa de integra��o. A placa de integra��o foi a �nica PCB desenvolvida neste projeto, sendo que ela reuni na mesma placa o conversor UART RS485, o conversor UART RS232 e ainda um conversor CC-CC chaveado que alimenta todo o console. O projeto do novo console culminou na montagem  do prot�tipo, o qual recebeu o nome de SICON III. Este prot�tipo montado em uma caixa de pl�stico adequada a instala��o do console no painel do autocarro, se apresentou como um produto com bom acabamento e bem montado. Quando testado o prot�tipo funcionou perfeitamente, realizando a configura��o do painel de teste e afixando adequadamente mensagens no mesmo. Ao fim foi desenvolvido um bom produto, dentro das necessidade e al�m das expectativas, e que possivelmente deve se tornar a pr�xima novidade na linha de indicadores de destino da DMS.
